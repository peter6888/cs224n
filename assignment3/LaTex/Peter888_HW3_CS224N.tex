\documentclass[fleqn]{MJD}

\usepackage{cancel}
\usepackage{cleveref}
\usepackage{titlesec}
\usepackage{hyperref}
%\colorsections
%\bluelinks
\newcommand{\problem}[1]{\chapter{Problem #1}}
\newcommand{\subproblem}[2]{\section{(#1)~ #2}}
\newcommand{\subsubproblem}[2]{\subsection{ #1)~ #2}}
\newcommand{\U}{\cup}
\renewcommand{\S}{\mathcal{S}}
\renewcommand{\s}{\subset}
\renewcommand{\equiv}{\Leftrightarrow}
\newcommand{\0}{\emptyset}
\newcommand{\imp}{\Rightarrow}
\newcommand{\Usum}{\bigcup\limits_{i=1}^\infty}
\newcommand{\intsum}{\bigcup\limits_{i=1}^\infty}
\newcommand{\infsum}{\sum\limits_{i=1}^\infty}
\newcommand{\sets}{\{A_1, A_2 \dots\} }
\newcommand{\nsets}{\{A_1, \dots, A_n \} }

\titleformat{\chapter}[display]
  {\normalfont\bfseries}{}{0pt}{\LARGE}
  
\graphicspath{ {../} }

%%%%%%%%%%%%%%%%%%%%%%%%%%%%%%%%%%%%
\begin{document}
\lstset{language=Python}
\titleAT[CS 224N: Assignment 3]{Peter888@stanford.edu}
%-------------------------------------
\problem{1. A window into NER (30 points)}
%-------------------------------------

%----------------------
\subproblem{a}{ Understanding NER (5 points, written)}
\subsubproblem{i} {Ambiguous Examples (2 points)} 
\noindent\textbf{Answer:} \\
\subsubproblem{ii} {Why use features (1 point)} 
\noindent\textbf{Answer:} \\
\subsubproblem{iii} {Describe the features (2 points)} 
\noindent\textbf{Answer:} \\

\vskip3em

%----------------------
\subproblem{b}{ Computational complexity (5 points, written)}
\subsubproblem{i} {Dimensions (2 points)} 
\noindent\textbf{Answer:} \\
\subsubproblem{ii} {Complexity (3 point)} 
\noindent\textbf{Answer:} \\

\vskip3em

%----------------------
\subproblem{c}{Implement model(15 points, code)}
\vskip3em
\newpage
%-------------------------------------
\problem{2. Recurrent neural nets for NER (40 points)}
%-------------------------------------

%----------------------
\subproblem{a}{Computational complexity (4 points, written)}
\subsubproblem{i} {How many more (1 point)} 
\noindent\textbf{Answer:} \\
\subsubproblem{ii} {Complexity (3 point)} 
\noindent\textbf{Answer:} \\
\vskip3em
%----------------------
\subproblem{b}{$F_{1}$ score (2 points, written)}
\subsubproblem{i} {When CE cost and $F_{1}$ decreasing at same time (1 point)} 
\noindent\textbf{Answer:} \\
\subsubproblem{ii} {Why not $F_{1}$ (1 point)} 
\noindent\textbf{Answer:} \\

\vskip4em


\subproblem{c}{RNN cell (5 points, code)}
\vskip4em
%----------------------
\subproblem{d}{RNN model (8 points, code/written)}
\subsubproblem{i} { Loss and Gradient Update (3 points, written)} 
\noindent\textbf{Answer:} \\
\subsubproblem{ii} { (5 points, code)} 
\vskip4em
%----------------------

\subproblem{e}{More RNN model (12 points, code)}
\vskip4em

\subproblem{e}{Train RNN model (3 points, code)}
\vskip4em

\newpage
%-------------------------------------
\problem{3. Grooving with GRUs (30 points)}
%-------------------------------------

%----------------------
\subproblem{a}{ Modeling latching behavior (4 points, written)}
\subsubproblem{i} {RNN cell values (1 point)}
\noindent \textbf{Answer:} \\ \\
\subsubproblem{ii} {GRU cell values (3 points)}
\noindent \textbf{Answer:} \\
\vskip4em

%----------------------
\subproblem{b}{Modeling toggling behavior (6 points, written)}
\subsubproblem{i} {1D RNN (3 points)}
\noindent \textbf{Answer:} \\ \\
\subsubproblem{ii} {GRU cell values (3 points)}
\noindent \textbf{Answer:} \\
\vskip4em

%----------------------
\subproblem{c}{GRU cell (6 points, code)}
\vskip5em
%----------------------
\subproblem{d}{Learn dynamics (6 points, code)}
\vskip4em
%----------------------

\subproblem{e}{Analyze graphs (5 points, written)}
\noindent \textbf{Answer:}  \\
\vskip4em
%----------------------
\subproblem{f}{Train GRU (3 points, code)}

\end{document}
